\documentclass[11pt,a4paper,sans]{moderncv}        % possible options include font size ('10pt', '11pt' and '12pt'), paper size ('a4paper', 'letterpaper', 'a5paper', 'legalpaper', 'executivepaper' and 'landscape') and font family ('sans' and 'roman')

\usepackage[document]{ragged2e}

% moderncv themes
\moderncvstyle{banking}                            % style options are 'casual' (default), 'classic', 'oldstyle' and 'banking'
\moderncvcolor{red}                                % color options 'blue' (default), 'orange', 'green', 'red', 'purple', 'grey' and 'black'
\renewcommand{\familydefault}{\rmdefault}         % to set the default font; use '\sfdefault' for the default sans serif font, '\rmdefault' for the default roman one, or any tex font name
%\nopagenumbers{}                                  % uncomment to suppress automatic page numbering for CVs longer than one page

% character encoding
\usepackage[utf8]{inputenc}                       % if you are not using xelatex ou lualatex, replace by the encoding you are using
%\usepackage{CJKutf8}                              % if you need to use CJK to typeset your resume in Chinese, Japanese or Korean

% adjust the page margins
\usepackage[scale=0.75]{geometry}

\firstname{}
\lastname{}
\begin{document}
%-----       letter       ---------------------------------------------------------

% recipient data
\recipient{Editors PLoS ONE}{}
\date{May 12th 2022}
\opening{Dear Editors,}
\closing{Yours faithfully,\\
Juste Raimbault\\
LASTIG, Univ Gustave Eiffel, IGN-ENSG
}
         % use an optional argument to use a string other than "Enclosure", or redefine \enclname
\makelettertitle

\justify
We are pleased, with my co-authors Natalia Zdanowska (Luxembourg Institute of Socio-economic Research) and Elsa Arcaute (University College London), to submit an original research article entitled ``Modeling growth of urban firm networks'' for consideration for publication in PLoS ONE. This article introduces a generative model for urban firm networks, to understand underlying processes, with possible applications to public policies. %It does not contain previously published work. Preliminary results were presented as a satellite abstract at the Conference on Complex Systems 2019 (September 2019) and will be presented as a main track abstract at the NetScience 2020 conference (September 2020). We did not have any previous interaction with PLoS regarding this manuscript.

A previous version of this manuscript was submitted to PLoS ONE in September 2020, with the submission identifier PONE-S-20-35765. Due to diverse unforeseen circumstances including job uncertainty and instability through the pandemic, we were not able to revise the manuscript in time after receiving a first round of reviews suggesting a revision. This current submission thus corresponds to the revision of this previous submission, and we included the response to reviewers and manuscript with track changes in submitted files.

This manuscript has not been published and is not under consideration for publication elsewhere, nor contains previously published work. A preprint of a previous version was deposited on arxiv with the identifier 2009.05528.

We have no conflicts of interest to disclose. We do not have any opposed reviewer.

%This paper was initially submitted to the PloS ONE collection ``Cities as Complex Systems'' edited by M. Gonzalez (University of California, Berkeley) and D. Rybski (Potsdam Institute for Climate Impact Research). We know that the official deadline has passed, nevertheless, we hope that it could still be considered for a non-special issue.


Thank you for your consideration.
\justify




\makeletterclosing





\end{document}


%% end of file `template.tex'.