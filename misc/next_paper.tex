\documentclass{article}


\begin{document}

% possible target Journal of Economic Geography? https://academic.oup.com/joeg/
% there is also Economic Geography- % https://www.tandfonline.com/toc/recg20/95/4?nav=tocList
%the editor is Jose Rodriguez-Pose from LSE, I just spoke to him at RSA conf, maybe it will be less econometric than JEG





%\citep{martinus2018global} combination of different types of proximity in inter-urban firm networks: economic, sociocultural, geopolitical - implies spatial and non-spatial 
%\citep{pan2017mapping} different types of linkages: e.g. services here, but not only. All sectors should be taken into account first.
%\cite{martinus2019brokerage} role of small countries
%\cite{dawley2019creating} regional policies for insertion in global network
%\cite{gluckler2016relational} positioning within networks



\section{Empirical analysis}

% Data sources:
%\begin{itemize}
%	\item AMADEUS database for company links
%	\item GHS UCDB for GDP of urban built-up areas
%	\item Functional urban areas (JRC-EC) for consistent ontology and spatial join between the different sources
%	\item Cities GDP can also be computed by aggregating company data: correlation between this approach and the GHS one must be checked: OK
%\end{itemize}


% TODO stat models justifying processes included:
% - poisson regression if we count a number of links?
% - country couples fixed effect to estimate the country distance matrix (as a function of interaction range ?)
% - multilevel modeling ?
% - distribution and properties (spatial stats ?) of sector distributions and similarities


\section{Model}

%\item Economic structure of cities are evolved with a city interaction model which can be
%    \begin{itemize}
%        \item a direct interaction model, according to
%        \begin{equation}
%            \Delta E_i = g_0\cdot E_i + w_G \cdot \sum_j V_{ij}/<V_{ij}>
%        \end{equation}
%        \item the Marius model with economic exchanges
%    \end{itemize}
%    \textit{Baseline: no evolution - on short time spans may be relevant ?}

% city size (taking not population but an economic variable related to firms avoids adding the dimension of population, simplifies the model) $\rightarrow$ can be combined with geographical distance with a classical interaction model, see \cite{cottineau2015growing}, \cite{raimbault2018indirect}, \cite{favaro2011gibrat}

% country boundaries are constructed to obtain an approximately equal coverage of the space for a fixed number of countries, what   remains consistent with non-correlated city sizes at the scale of the countries as our cities are randomly distributed in space \citep{simini2019testing}(\textit{Note: on this point this may be realistic if staying at a continental level, although we should include realistic size of countries; this paper is surely wrong when changing the dataset used and the scale of analysis, working at the scale of Mega-city regions e.g.}). Therefore, we do a basic spatial k-means clustering with $C = 20$ countries and attribute the country accordingly.

% firm sector structure: 
%        \item random probabilities
%        \item uniform: then the term of proximity is cancelled, no interest
%        \item function of size:
% function of size: larger cities are more diverse, larger cities have more knowledge-based sectors. Test 1: log-normal with slightly increasing width and shifting mode.
        % mode of log normal is exp(mu - sigma ^ 2) ; variance is [exp(sigma^2) - 1]*exp(2 mu + sigma ^ 2)
        % when the pdf is 1 / (x sigma sqrt(2 pi)) * exp (- (ln x - mu) ^ 2 / 2 sigma^2 )
        % -> for largest city stddev = sqrt(variance) = fields / 2 ; for smallest sigma = 1 / nfields
        %  ==> solving for mu and sigma with linear mode and variance as a function of log(E_i)
        % => sigma^2 is the unique positive root of f(X)=0 with f(X) = -3X - 2 ln(exp(X)-1) - ln(e_i)
        % => mu = sigma^2 + ln(e_i)
        %    ~ somehow dirty and not sure that most of the distrib for highest values is within [0,1]





% \textit{rq: as these are done sequentially, it is equivalent to have a fixed time step and a number of linkages per time step, or to do it link by link - in the data-driven version the actual values shall be parametrised, while in the synthetic version this has no importance}
% -> No because of path dep

% TODO the path dependency $w_{ij}$ could (or should ?) be decomposed into in/out degree for origin and destination ? (then stronger self-reinforcment effect)

% These competing drivers are combined - how ? \textit{Options: (i) multiplicative function to get probabilities; (ii) generalization with a multi-attribute Cobb-Douglas; (iii) multi-objective optimization depending on sectors, using Pareto fronts ? - needs empirical evidence; (iv) copula-based combination as done by \cite{2019arXiv190505106C} - copula parameters are free or estimated from real data ?}


% Cosine similarity
% (\textit{should be amended/validated/tested with an asymmetric function in a further version})

% ! links depend on the sectors: cobb douglas for sector proximity should depend on sector composition ? or several functions ?



% possible other indics:
% weighted degree distribution (\textit{if it is fat-tailed, log-log power law exponent and adjusted r-squared})
%Sector structure indicators: sector diversity (Herfindhal ?) 

%Two other thematic indicators, which are regionalisation (correlation between length and flow of links, stratified by size of extremities), and specialisation (correlation between sector proximity and flow of links, stratified by size of extremities), are not included in this study for the sake of simplicity as each has a high dimension.


\section{Data}

Datasets to explore/compare/validate: https://www.lboro.ac.uk/gawc/data.html


%%%
% with new dataset we do not have a subsample of companies within a specific country anymore
%More probably, we may have to work on an extract of the AMADEUS database including only links from and to UK-based companies. This constraints imposes modifications on the level to remain consistent:
%\begin{itemize}
%    \item the adjustment error can only be partial and computed on the UK links only
%    \item an initialization method for the missing links must be implemented (e.g. through a statistical analysis? not far from the model then)
%    \item the model can be run without initial links, and an additional parameter sets the duration necessary to reach the time considered as the first date
%\end{itemize}

%If the links are ``complete'' within Europe, the model (i) can be initialized at $t_0$ with the network corresponding to the first date in the dataset; (ii) adjustment error is computed on the later dates.
% -> pb of dates - can not have a truly dynamical calibration



\begin{table}
    \caption{List of parameters, associated processes, and ranges.}
    \begin{center}
    \begin{tabular}{|c|c|c|c|}
    \hline
    Parameter & Name & Process & Range \\\hline
      $\gamma_F$ & Origin &  & $\left[0;10\right]$ \\
    $\gamma_T$ & Destination & & $\left[0;10\right]$ \\
    $\gamma_W$ & Previous links & Self-reinforcement & $\left[0;10\right]$  \\
    $\gamma_D = \frac{1}{d_G}$ & Geographical distance & & $\left[0;10000\right]$ \\
    $\gamma_G  = \frac{1}{d_G}$ & Socio-cultural distance & & $\left[0;10000\right]$ \\\hline
    \end{tabular}
    \end{center}
\end{table}




\section{Results}

% Implementation
% Netlogo for vis + scala into spatialdata / urbangrowth ?

% On vine copulas :
%    https://link.springer.com/article/10.1023/A:1016725902970
%    https://www.groups.ma.tum.de/en/statistics/research/vine-copula-models/
%    R package : https://github.com/tnagler/VineCopula
% Q: what would be the meaning on parametrizing interaction function with a real Vine copula? non-linear interactions? (on log, linear combination of factors)

%We use in particular data structures providing good performances for the type of operations needed (matrices operations for the construction of the utility function e.g.).


%\todo{explore effect of $t_f$ and/or $w_0$; be closer to real setup?} % -> more interesting to explore effect of urban hierarchy


%\subsubsection{PSE}

%\subsubsection{A virtual case study: subsystem-xit}

% Q: do we do some "hybrid experiments"? - ex synthetic system of cities but real sector distribution


\section{Discussion}


% Several scenarios are considered taking into consideration the current and future economic challenges, such as those given by Brexit. The model may be applied in future work to assess policy issues and recommendations.

% - other formulation of the combination of factors
%multi-objective optimisation depending on sectors, using Pareto fronts ? - needs empirical evidence; (iv) copula-based combination as done by \cite{2019arXiv190505106C} - copula parameters are free or estimated from real data
% The function used can be understood as a separable spatial interaction model. As a possible extension, Copula capture a dependency structure but must be parametrised by estimation on real data.

% - evolution of city sizes (co-evolution model)
% - role of path dependency
% - towards a model with firm agents? (multi-scale ABM)






\end{document}
