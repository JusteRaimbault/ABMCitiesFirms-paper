\input{si.tex}

\renewcommand{\thetable}{S2-\Alph{table}}

\begin{document}
\vspace*{0.35in}
\justify

\section*{S2 Text: Statistical models, generative model setup and sensitivity analysis}


\subsection*{Socio-economic distance matrix}

In the real model setup, the socio-economic distance between countries $c_{ij}$ is constructed using the fixed effects coefficients from the statistical model. In Table~1 in the main text, we estimated several statistical models, including some with fixed effects by pairs of countries. The socio-cultural distance between two countries is then taken as the opposite of the fixed effect coefficient for this pair, for the full statistical model. The summary statistics for the 29x29 matrix (28 EU countries and Jersey, which is in the database corresponds to a distinct country from the UK, although it has an important total turnover since many British companies are located there for fiscal advantage reasons) are shown below in the Table~\ref{tab:fixedeff}. We observe that distances are rather localised but with large outliers, and an important proportion are not defined (339 out of 841), corresponding to a couple of countries having no exchange at all in the dataset.

\begin{table}[h!]
\caption{\textbf{Summary statistics of the socio-cultural distance estimated with a fixed effects model.}\label{tab:fixedeff}}
\begin{tabular}{ccccccc}
   Min. & 1st Qu. &  Median &  Mean &  3rd Qu. &  Max. &  NA's \\
 -2.327 &  1.440 &  2.451 &  2.538 &  3.447 & 11.797 &  339 
\end{tabular}
\end{table}



\subsection*{Zero-inflated models}

We show in Table~\ref{tab:zeroinfl} estimation results for zero-inflated and hurdle statistical models. We only show models with all variables included, which had the best AIC values. Main model coefficients are very close in both approaches, and do not deviate largely from values obtained with the simpler model. All coefficients have a high significance in this case. Regarding the additional components, both give the same qualitative outcome (with opposite signs, the hurdle interpretation being the opposite of the zero-inflated). Distance increase the likelihood of having a zero flow, while other variables decrease it, with the same order of magnitude of effect size than the main component. R-squared values are much lower than simpler model, since a large proportion of the dataset are zeros and the models are thus harder to fit when including them.


%%%%%%%%%%%%%
\begin{table}[!ht]
\begin{adjustwidth}{-1in}{0in}
\caption{{\bf Zero-inflated and Hurdle statistical models.}\label{tab:zeroinfl}}
\medskip
\begin{tabular}{|l|c|c|c|c|}
\hline
Model  & \multicolumn{2}{|c|}{Zero-inflated} & \multicolumn{2}{|c|}{Hurdle} \\ 
 & Count & Zero-infl. & Count & Zero-hurdle \\
\hline
$\log(d_{ij})$ &    -0.28*** (7e-4) &   0.88*** (0.01)      &   -0.29*** (8e-6)   &  -0.75*** (0.01)    \\
$\log(T_i)$ & 0.74*** (4e-6)  & -0.74*** (7e-3)       &   0.74***  (5e-6)     &  0.67*** (5e-3)   \\
$\log(T_j)$ & 0.64*** (4e-6)  & -0.52*** (7e-3)      &    0.64***  (4e-6)     &  0.48*** (5e-3)    \\
$\log(s_{ij})$ & 0.46*** (2e-5)   &  -0.33*** (3e-2)  &  0.45*** (2e-5)       &  0.27*** (0.02)    \\
\hline
$R^2$ &   \multicolumn{2}{|c|}{0.1607}     &  \multicolumn{2}{|c|}{0.1661}  \\
%MSE & & \\
AIC &     \multicolumn{2}{|c|}{5.85e10}    &   \multicolumn{2}{|c|}{5.85e10}  \\
\hline
\end{tabular}
\end{adjustwidth}
\end{table}
%%%%%%%%%%%%%



\subsection*{Synthetic model setup}

The procedure for the synthetic sector setup is the following:
\begin{itemize}
    \item Sectors distributions follow log-normal densities with most mass in $\left[0;1\right]$
    \item Large cities are more innovative and more diverse. Assuming that sectors are ordered by innovativity (the larger the sector index the larger the innovativity), this assumption is translated by taking a mode and variance of $1/2$ for the sector distribution of the largest city and a mode and variance of $1/K$ for the sector distribution of the smallest city. Intermediate cities have their sector distribution mode and variance following a linear interpolation between the two. For each city, we therefore define this value for mode and variance as $e_i$ by taking
    \[
    e_i = \frac{(\log(E_i) - \log(E_{imin}))}{(\log(E_{imax}) - \log(E_{imin}))} \cdot (1/2 - 1/K) + 1/K
    \]
    where $E_i$ is the economic size of city $i$, $E_{imin}$ size of smallest city and $E_{imax}$ size of the largest city;
    \item Log-normal parameters $(\mu_i,\sigma_i)$ for each city are then determined by using the formula for the mode and variance of a log-normal law; the mode is directly given by $\mu_i - \sigma_i^2 = \log(e_i)$, while the variance verifies $\log(\exp(\sigma_i^2) - 1) + 2\mu + \sigma^2  = log(e_i)$; using the first equation to have an expression for $\mu$ and substituting into the second, we obtain
    \[
    -3 \sigma_i^2 - 2 \log(\exp(\sigma_i^2) - 1) = log(e_i)
    \]
    \item $\sigma_i^2$ is thus obtained as the unique positive root of $f(X)=0$ with $f(X) = -3X - 2 \log(\exp(X) - 1) - \log(e_i)$ (this function has a derivative always negative on $\mathbb{R}+$, tends to $\infty$ when $X \rightarrow 0+$ and to $-\infty$ when $X \rightarrow \infty$); we finally compute $\mu_i$ as $\mu_i = \log(e_i) + \sigma_i^2$.
\end{itemize}




\subsection*{Sensitivity analysis}

The table~\ref{tab:saltelli} gives the full results for the Global Sensitivity Analysis, for all model indicators and free parameters. We give the first order indices and the total indices.

%%%%%%%%%%%%%
\begin{table}[h!]
\begin{adjustwidth}{-2.25in}{0in}
\caption{Saltelli sensitivity indices, for indicators in rows and parameters in columns. We give for each pair the first order index (F) and the total order index (T).\label{tab:saltelli}}
\hspace{-1cm}\begin{tabular}{|l|c|c|c|c|c|c|c|c|c|c|c|c|}
\hline
 & \multicolumn{2}{|c|}{$d_0$} & \multicolumn{2}{|c|}{$c_0$} & \multicolumn{2}{|c|}{$\gamma_S$} & \multicolumn{2}{|c|}{$\gamma_W$} & \multicolumn{2}{|c|}{$\gamma_O$} & \multicolumn{2}{|c|}{$\gamma_D$} \\
 & F & T & F & T & F & T & F & T & F & T & F & T \\
 \hline
Internationalisation & 0.2 & 0.3 & 0.7 & 0.7 & 0.001 & 0.008 & $5\cdot 10^{-4}$ & 0.007 & 0.03 & 0.04 & 0.02 & 0.04 \\
Metropolisation & 0.02 & 0.04 & 0.02 & 0.04 & 0.02 & 0.3 & 0.002 & 0.01 & 0.2 & 0.7 & 0.2 & 0.7 \\
Modularity & 0.3 & 0.4 & 0.6 & 0.6 & $5\cdot 10^{-4}$ & 0.02 & 0.002 & 0.02 & 0.007 & 0.03 & 0.003 & 0.03 \\
Avg. com. size & 0.004 & 0.1 & 0.01 & 0.1 & 0.003 & 0.07 & 0.001 & 0.04 & 0.3 & 0.6 & 0.3 & 0.6 \\
Degree entropy & 0.003 & 0.03 & 0.002 & 0.03 & 0.008 & 0.04 & 0.003 & 0.03 & 0.5 & 0.6 & 0.5 & 0.6 \\
Weight entropy & 0.04 & 0.2 & 0.03 & 0.2 & 0.002 & 0.1 & 0.001 & 0.1 & 0.5 & 0.6 & 0.5 & 0.6 \\\hline
\end{tabular}
\end{adjustwidth}
\end{table}
%%%%%%%%%%%%%





\end{document}