\documentclass[10pt,letterpaper]{article}
\usepackage[margin=1in]{geometry}
%\usepackage{natbib}
\usepackage{multicol}
\usepackage{graphicx}
\pagenumbering{gobble}
\usepackage{color}

\newcommand{\ea}[1]{{\color{blue}  #1}}

\title{\vspace{-2.5cm}A generative model for urban firm networks}
\author{J. Raimbault$^{1,2,3}$, N. Zdanowska$^{1,3}$ and E. Arcaute$^1$\\\medskip\small
$^{1}$ Centre for Advanced Spatial Analysis, UCL; $^{2}$ UPS CNRS 3611 ISC-PIF; $^{3}$ UMR CNRS 8504 G{\'e}ographie-cit{\'e}s
}
\date{}

\begin{document}

\maketitle

\vspace{-0.5cm}

The emergence of a global network of cities, defined by interactions between firms, is at the heart of recent globalization processes \cite{taylor2001specification}. Understanding growth mechanisms of such a network is crucial to foster innovation, shape local economy policies \cite{turkina2016structure}, and global industrial strategies for regions \cite{dawley2019creating}, cities \cite{gluckler2016relational} and countries \cite{martinus2019brokerage}. Urban networks are characterized by interactions of different nature and determined by economic, socio-cultural, or geopolitical proximity \cite{martinus2018global}. This paper focuses on economic relations between urban areas defined by ownership links of firms. Since transnational firms structure is one of the drivers of the global economic space, it is one of the proxies to unveil geographical structures \cite{2019arXiv191014652Z}. 

This work introduces a generative network model to simulate the growth of such links at urban areas scale within the evolutionary systems of cities framework \cite{pumain2018evolutionary}. We combine multiple factors which influence link formation, as economic intensity of the origin and destination of urban areas, industrial sector proximity between cities, the strength of links from the past, and geographical and socio-cultural distance. The model allows thus to compare the effect of different factors on the final network structure, and to extrapolate their actual effect when calibrated on real data. Formally, the model creates links between cities which are characterized by their economic size $E_i$ (GDP) and economic structure $S_{ik}$ in terms of activity sectors (probability distribution of firms within $K$ sectors). Starting with an existing network with no links, the model iteratively adds links, following a probability given by a generalized Cobb-Douglas function \cite{vilcu2011geometric} as 
\[
p_{ij} \propto \left(\frac{E_{i}}{E}\right)^{\gamma_F} \cdot \left(\frac{E_{j}}{E}\right)^{\gamma_T} \cdot \left(\frac{w_{ij}}{W}\right)^{\gamma_W} \cdot s\left(S_{ik},S_{jk}\right)^{\gamma_S} \cdot \exp \left(- \gamma_G \cdot d_{ij}\right) \cdot \exp \left(- \gamma_D \cdot g_{ij}\right)
\]
where $E  =  \sum_k E_k$, $W  = \sum_{i,j} w_{ij}$ weights of previous links, $s(S_{ik},S_{jk})$ is similarity function between activity sectors $S_{ik}$ and $S_{jk}$, $d_{ij}$ euclidian distance, and $g_{ij}$ a socio-cultural distance.

We consider several macroscopic indicators to quantify the generated network. Firstly, geographical structures are captured by (i) internationalisation (modularity of countries in the network); (ii) metropolisation (correlation between weighted degree and city size); (iii) regionalisation (correlation between length and flow of links, stratified by size of extremities); and (iv) Specialisation (correlation between sector proximity and flow of links, stratified by size of extremities). We also consider several network indicators as Louvain modularity, community sizes, degree and link weight distributions (described by their average, hierarchy, and entropy), and correlations between degree and city size, and between link weight and distance.

The model is implemented in NetLogo and integrated into the OpenMOLE platform for model exploration to distribute the computation on high performance infrastructures. We build the model on a synthetic system of cities, at the scale of a continent with properties similar to Europe: (i) 700 cities with an empirical power-law from the Global Human Settlements database for economic sizes; (ii) clustered into 30 countries; (iii) with synthetic distribution of sectors following log-normal distributions adjusted in a way that large cities have more high-value activities and are more diverse. 

As a matter of validation, the model is compared with real data for Europe : Functional Urban Areas from the Global Human Settlement Database for cities and their characteristics, and the AMADEUS database produced by the Bureau Van Dijk on ownership links between firms in Europe between 2010 and 2018 at postcode level in all types of sectors. Calibration is done using a genetic NSGA2 algorithm, finding the parameter values giving network trajectories closest to reality. First results show in particular a role of the path-dependency parameter confirming the relevance of this complex generative model. Further work is needed, although our study demonstrates the feasibility of model calibration on real data. Several scenarios are considered taking into consideration the current and future economic challenges, such as those given by Brexit. The model may be applied in future work to assess policy issues and recommendations.

%\begin{figure}
%\vspace{-2cm}
%\begin{center}
%    \begin{minipage}[c]{0.25\textwidth}
%        \includegraphics[width=\textwidth]{figures/ex_alleq-lowgravity_seed-12102_t1500.png}\\
%        \includegraphics[width=\textwidth]{figures/ex_alleq-highgravity_seed-12102_t1500.png}
%    \end{minipage}
%    \begin{minipage}[c]{0.6\linewidth}
%     \includegraphics[width=0.48\textwidth]{figures/internationalization-gravityDecay_errorbars.pn%g}
%    \includegraphics[width=0.48\textwidth]{figures/rhoDegreeSize-gravityDecay_errorbars.png}\\
%    \includegraphics[width=\textwidth]{figures/networkAvgCommunitySize_countryGravityDecay2100_gam%maDestination0_facetwrapgammaOrigin_colorgammaSectors.png}
%    \end{minipage}
%    \end{center}
%    \vspace{-0.5cm}
%    \caption{\footnotesize\textbf{(Left column)} Example of simulated networks for a synthetic %system of cities, with low gravity range (top) and high gravity range (bottom); \textbf{(Top %right)} Internationalization and correlation between size and degree as a function of decay %parameter; \textbf{(Bottom right)} Average network community size as a function of decay %parameter, for varying influence of sectors ($\gamma_S$ in color) and of origin ($\gamma_F$ in %columns).}
%\end{figure}

\footnotesize


\begin{multicols}{2}

\bibliographystyle{unsrt}
\bibliography{biblio}

\end{multicols}

\end{document}